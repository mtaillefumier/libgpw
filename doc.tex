\documentclass[prb]{revtex4}
\usepackage{amsmath}
\begin{document}

We want to compute the density matrix starting of a quantum state
$\left|\phi\right>$ decomposed on a gaussian basis set (not necessary orthogonal
basis). the over of the basis set is regrouped in product of two gaussians with
parameter $\eta_1$ and $\eta_2$ that are centered around two points ${\bf r}_1$
and ${\bf r}_2$ in real space. Moreover the summation involves sum over angular
momentums $l_i\in\left[l_{\sf min}^i, l_{\sf max}^i\right]$ with $i=1,2$ ($i$ is
not a exponent here). The contribution of this group of gaussians multiplied by
the cartesians harmonics to the density is given by
\begin{equation}
  \label{eq:1}
  n(x,y,z) = \sum_{l_1 = l_{min}^1}^{l_{max}^1} \sum_{l_1 = l_{min}^2}^{l_{max}^2} \sum_{\substack{\alpha_1 + \beta_1 + \gamma_1 = l_1\\\alpha_2 + \beta_2 + \gamma_2 = l_2}} C^{l_1,l_2}_{\alpha\beta\gamma} (x-x_1)^{\alpha_1} (x-x_2)^{\alpha_2}(y-y_1)^{\beta_1} (y-y_2)^{\beta_2}(z-z_1)^{\gamma_1} (z-z_2)^{\gamma_2} \exp_a\exp_b
\end{equation}
where $C^{l_1,l_2}_{\alpha\beta\gamma}$ and the weights of each individual
gaussian in $\left|\phi\right>$ and $\alpha=(\alpha_1,\alpha_2)$,
$\beta=(\beta_1,\beta_2)$, $\gamma=(\gamma_1,\gamma_2)$. The exponents
$\alpha_i,\beta_i,\beta_i$ are constrained by the following relations
\begin{eqnarray}
  l1 &=& \alpha_1 + \beta_1 + \gamma_1\\
  l2 &=& \alpha_2 + \beta_2 + \gamma_2
\end{eqnarray}
The number of solutions is given by $1, 3, 6, 10, 15, 21, 28, 36, 45, 55, 66$
for $l_1 = 0, 1, 2, 3, 4, 5, 6, 7, 8, 9, 10$.

Evaluating Eq.\ref{eq:1} for a fixed $l_1 = 2$, $l_2 = 3$ for instance requires
$60$ possible values of $(\alpha_1,\beta_1,\gamma_1,\alpha_2,\beta_2,\gamma_2)$
(fma operations) times the number of grid points we need to evaluate the
polynomial (periodic boundaries conditions are neglected here). Although it is
easily parallizable, it requires 60 accesses to the density grid. If $l_1$,
$l_2$ covers an interval, all of this adds up which means it is unpractical to
do the summation the brute force way.

\section{Orthorombic lattices}
One method to compute Eq.\ref{eq:1} is to use the fact that the
product of two gaussians is a gaussian of width $\eta_{12} =
\eta_1\eta_2/(\eta_1 + \eta_2)$ centered around ${\bf r}_{12} = (\eta_1 {\bf
  r}_1 + \eta_2 {\bf r}_2)/ (\eta_1 + \eta_2)$ with a weight that depends only
on $\eta_1,\eta_2,{\bf r}_1,{\bf r}_2$.

Instead of evaluating Eq.1 the brute force way, we express the product of
polynomials $(x-x_1)^{\alpha_1} (x-x_2)^{\alpha_2}$ as
\begin{equation}
  (x-x_1)^{\alpha_1} (x-x_2)^{\alpha_2} = \sum_{k}^{\alpha_1}\sum_{k'}^{\alpha_2}
  \left(
  \begin{array}{c}
    \alpha_1\\
    k
  \end{array}
  \right)
  \left(
  \begin{array}{c}
    \alpha_2\\
    k'
  \end{array}
  \right)
  (x-x_{12})^k (x_{12} - x_1)^{\alpha_1 - k}  (x_{12} - x_2)^{\alpha_2 - k'}
\end{equation}
where $\left(
\begin{array}{c}
  n\\
  k
\end{array}
\right)$ is the binomial coefficient.

If we apply this to Eq.\ref{eq:1} in all three directions, we obtain
\begin{eqnarray}
  \label{eq:2}
  n(x,y,z)  &=&
            \sum_{l_1 = l_{\sf min}^1}^{l_{\sf max}^1} \sum_{l_1 = l_{\sf min}^2}^{l_{\sf max}^2} \sum_{\alpha_1,\alpha_2}^{l_1,l_2} \sum_{\substack{\beta_1 = 0\\\beta_2 = 0}} ^{\substack{l_1 - \alpha_1\\l_2 - \alpha_2}}  C^{l_1,l_2}_{\alpha\beta\gamma} \sum_{k_0} ^ {\alpha_1}\sum_{k_1} ^ {\alpha_2} \left(
            \begin{array}{c}
              {\alpha_1}\\
              k_0\end{array}
              \right)
              \left(
              \begin{array}{c}
                \alpha_2\\
                k_1
              \end{array}\right)
              \left(\begin{array}{c}
                \beta_1\\
                k_2\end{array}
                \right)
                \left(
                \begin{array}{c}
                  \beta_2\\
                  k_3
                \end{array}\right)
                \left(\begin{array}{c}
                  l_1 - \beta_1 - \alpha_1\\
                  k_4\end{array}
                  \right)
                  \left(
                  \begin{array}{c}
                    l_2 - \beta_2 - \alpha_2\\
                    k_5
                  \end{array}\right)\nonumber\\ &\times&
                  (x-x_{ab})^{k_0+k_1} (y-y_{ab})^{k_2+k_3} (z - z_{ab}) ^ {k_4 + k_5}\nonumber\\ &\times& (z_{ab} - z_a) ^{l_1 - \beta_1 - \alpha_1 - k_4} (z_{ab} - z_b)^{l_2 - \alpha_2 - \beta_2-k_5}\nonumber  \\ &\times& (y_{ab} - y_a) ^{\beta_1 - k_2} (y_{ab} - y_b)^{\beta_2-k_3}\nonumber \\ &\times& (x_{ab}-x_a)^{\alpha_1 - k} (x_{ab}-x_b)^{\alpha_2 - k'} \exp_a\exp_b
\end{eqnarray}
The advantage is that Eq.\ref{eq:2} only contains monomials of the form
$(x-x_{12}) (y-y_{12}) (z-z_{12})$ with possibly different powers. For instance
for $l_1 = 2$, $l_2 = 3$ (only one pair) we have $46$ monomials instead of $60$.

Now if $l_1, l_2$ covers some range $l_1\in \left[0, 2\right]$, $l_2\in \left[0,
  3\right]$, we obtain an expression with the same number of monomials instead
of $200$. We already have a factor 4 at least between this approach and the
brute force approach.

In discrete form and for the orthorombic case, Eq.\ref{eq:2} can be written as a
matrix-matrix product (actually a tensor - tensor product)
\begin{equation}
  \label{eq:3}
  n_{ijk} = \sum_{l = 0}^{l^1_{\sf max} + l^2_{\sf max}}  A_{\alpha\beta\gamma} X_{\alpha i} Y_{\beta j} Z_{\gamma k},
\end{equation}
with
\begin{eqnarray}
  X_{i\alpha} &=& (x_i - x_{12})^\alpha \exp(-\eta_{12} (x_i-x_{12})^2)\\
  Y_{j\beta} &=& (y_j - x_{12})^\beta \exp(-\eta_{12} (y_j - y_{12})^2)\\
  Z_{k\gamma} &=& (z_k - z_{12})^\gamma \exp(-\eta_{12} (z_k-z_{12})^2),
\end{eqnarray}
with
\begin{eqnarray}
  \label{eq:8}
  A_{\alpha\beta\gamma} &=&\left.\frac{\partial^{\alpha+\beta+\gamma} n({\bf
      r})}{\partial^\alpha x \partial ^ \beta y \partial ^ \gamma z} (x - x_{12})^\alpha (y - y_{12})^\beta (z_k -z_{12})^\gamma\right|_{r=r_{12}},\\
  &=&  \sum_{l_1 = l_{\sf min}^1}^{l_{\sf max}^1} \sum_{l_1 = l_{\sf min}^2}^{l_{\sf max}^2} \sum_{\alpha_1 + \alpha_2 = \alpha}^{\substack{\alpha_1 \leq l_1\\ \alpha_2 \leq l_2}} \sum_{\substack{\beta_1 = 0\\\beta_2 = 0}} ^{\substack{l_1 - \alpha_1\\l_2 - \alpha_2}}  C^{l_1,l_2}_{\alpha\beta\gamma} \sum_{k_0} ^ {\alpha_1}\sum_{k_1} ^ {\alpha_2} \left(
            \begin{array}{c}
              {\alpha_1}\\
              k_0\end{array}
              \right)
              \left(
              \begin{array}{c}
                \alpha_2\\
                k_1
              \end{array}\right)
              \left(\begin{array}{c}
                \beta_1\\
                k_2\end{array}
                \right)
                \left(
                \begin{array}{c}
                  \beta_2\\
                  k_3
                \end{array}\right)
                \left(\begin{array}{c}
                  l_1 - \beta_1 - \alpha_1\\
                  k_4\end{array}
                  \right)
                  \left(
                  \begin{array}{c}
                    l_2 - \beta_2 - \alpha_2\\
                    k_5
                  \end{array}\right)\nonumber\\ &\times&
                  (y_j-y_{ab})^{k_2+k_3} (z_k - z_{ab}) ^ {k_4 + k_5}\nonumber\\ &\times& (z_{ab} - z_a) ^{l_1 - \beta_1 - \alpha_1 - k_4} (z_{ab} - z_b)^{l_2 - \alpha_2 - \beta_2-k_5}\nonumber  \\ &\times& (y_{ab} - y_a) ^{\beta_1 - k_2} (y_{ab} - y_b)^{\beta_2-k_3}\nonumber \\ &\times& (x_{ab}-x_a)^{\alpha_1 - k} (x_{ab}-x_b)^{\alpha_2 - k'} \exp_a\exp_b
\end{eqnarray}
taking all terms $k_1 + k_2 = \alpha$ $k_3 + k_4 = \beta$ and $k_5 + k_6 =
\beta$ with $k_1 \leq l^1_{\sf max}$ and $k_2 \leq l^2_{\sf max}$ in
Eq.\ref{eq:3}.

$n_{ijk}$ can be evaluated with tensor-tensor products which reduces to matrix -
matrix products. The matrix products are evaluated the following way
\begin{equation}
  T^1_{\alpha\beta k} = \sum_{\gamma} A_{(\alpha\beta)\gamma} Z_{\gamma, k}
\end{equation}
the $(\alpha\beta)$ notation indicating that we treat it a composite indice. Then
\begin{equation}
  T^2_{\alpha j k} = \sum_{\beta} T^1_{\alpha k \beta} Y_{\beta, j}
\end{equation}
where we need to transpose the second and third indices of $T^1$ and finally
\begin{equation}
  n_{ijk} = \sum_{\alpha} X_{\alpha, i} ^T T^2_{\alpha (j k)}
\end{equation}

\section{Non orthorombic lattices}
Evaluating the density operator on a non orthorombic lattice or grid requires a
priori to evaluate all polynomials for each point of the grid if we use
Eq.\ref{eq:1}. However, it is possible to recast Eq.\ref{eq:1} for non
orthorombic lattices into a problem that has a structure that is similar to the
orthorombic case.

We suppose the grid regular in the basis formed by the displacement vectors of
the lattice. We will write then as $v_1, v_2, v_3$. A grid point $ijk$ will have
the following cartesian coordinates
\begin{equation}
  \label{eq:6}
  {\bf r}_{ijk} = (x_i, y_j, z_k) = i {\bf v}_1 + j {\bf v}_2 + k {\bf v}_3
\end{equation}
$x_i, y_j, z_k$ can be obtained after projection along $x,y,z$. The point is
that we can replace the cartesian coordinates by their expression in
Eq.\ref{eq:2}. We will obtain formula of this type
\begin{equation}
  n_{ijk} = \sum_{\alpha\beta\gamma}  A_{\alpha\beta\gamma} I_{\alpha i} J_{\beta j} K_{\gamma k} {\sf Exp}_{ij} {\sf Exp}_{jk} {\sf Exp}_{ik},
\end{equation}
where
\begin{eqnarray}
  X_{i\alpha} &=& (i - x'_{12})^\alpha \exp(-\eta_{12} (i ^ 2 (v_1)^2 - x'_{12})^2)\\
  Y_{j\beta} &=& (j - y'_{12})^\beta \exp(-\eta_{12} (j ^ 2 (v_2)^2 - y'_{12})^2)\\
  Z_{k\gamma} &=& (k - z'_{12})^\gamma \exp(-\eta_{12} (k^2 (v_3)^2 - z'_{12})^2),
\end{eqnarray}
and
\begin{eqnarray}
  {\sf Exp}_{ij} &=& \exp\left(\eta_{12} (v_1 \cdot v_2) (i-x'_{12})(j-y'_{12}) \right)\\
  {\sf Exp}_{jk} &=& \exp\left(\eta_{12} (v_2 \cdot v_3) (j-y'_{12})(k-z'_{12}) \right)\\
  {\sf Exp}_{ik} &=& \exp\left(\eta_{12} (v_1 \cdot v_3) (i-x'_{12})(k-z'_{12}) \right).
\end{eqnarray}
If the boundaries conditions are open then it is possible to use the same trick
than for the orthorombic case and then multiply the final result with the
corrections ${\sf Exp}_{ij}$.

When period boundaries conditions are imposed, we have a extra summation over
the period in the dimensions wheren the PBC is imposed. $n_{ijk}$ ($i,j,k$ are
now within the unit cell) becomes
\begin{equation}
  n_{ijk} = \sum_{n,m,l} \left(\sum_{\alpha\beta\gamma}
  A_{\alpha\beta\gamma} I_{\alpha, i+m} J_{\beta, j+n} K_{\gamma, {k+l}} {\sf
    Exp}_{i+m, j+n} {\sf Exp}_{j+n, k+l} {\sf Exp}_{i + m, k + l}\right)
\end{equation}
summation over $\alpha\beta\gamma$ can still use the same method than for the
orthorombic case, but we have to repeat the same calculations multiple times to
cover the summation over $n,m,l$.
\end{document}
